\documentclass[11pt]{article}
\usepackage{fontspec}
\usepackage[utf8]{inputenc}
\setmainfont{Bodoni 72 Book}
\usepackage[paperwidth=11in,paperheight=17in,margin=1in,headheight=0.0in,footskip=0.5in,includehead,includefoot,portrait]{geometry}
\usepackage[absolute]{textpos}
\TPGrid[0.5in, 0.25in]{23}{24}
\parindent=0pt
\parskip=12pt
\usepackage{nopageno}
\usepackage{graphicx}
\graphicspath{ {./images/} }
\usepackage{amsmath}
\usepackage{tikz}
\newcommand*\circled[1]{\tikz[baseline=(char.base)]{
            \node[shape=circle,draw,inner sep=1pt] (char) {#1};}}

\begin{document}

\vspace*{4\baselineskip}

\begingroup
\begin{center}
\huge NOTES FOR THE INTERPRETERS
\end{center}
\endgroup

\begingroup
\textbf{General: 1.)} This work is to be played with an opaque curtain between the interpreters and the listeners, and with each instrument amplified. The listeners will be in the center of a nine-speaker ellipse. Each speaker will receive the signal of a single instrument. The ordering of the instruments, starting with the frontmost speaker and rotating clockwise, are as follows: mezzo-soprano, second cello, bass flute, English horn, percussion, tuba, violin, flute, first cello. \textbf{2.)} The poem following this page should be projected before the listeners as it is read, amplified and from behind the curtain, by the ensemble according to the script following the setting proper. In interpreting the script, lines which change speakers without a line break should flow as a complete sentence, and should not be broken when the speaker changes. Line breaks are to be understood as brief pauses. \textbf{3.)} If the work is performed after December 21st and before March 19th, the poem should be read before the music. If the work is performed after March 20th and before December 20th, the poem should be read after the music. \textbf{4.)}  Dynamics in this score are effort dynamics, representing the physical force behind an action rather than the sounding dynamic. \textbf{5.)} Dashed arrows indicate a gradual transition from one technique or tempo to another. \textbf{6.)}  Stem tremoli are to be performed as quickly as possible, and do not represent a subdivision of a note. \textbf{7.)} Flat glissandi are sometimes used for the same function as ties. \textbf{8.)} Empty measures are to be understood as full-measure rests. \textbf{9.)} Grace notes are to be performed before the beat they precede, as quickly as possible. \textbf{10.)} Accidentals apply only to the note which they immediately precede.
\endgroup

\begingroup
\textbf{Winds and Woodwinds: 1.)} In extended passages where no breaths or rests are notated, interpreters are encouraged to break the line at their discretion.
\endgroup

\begingroup
\textbf{Flutes: 1.) Head joint tilt} is represented by degree articulations, wherein \textbf{0°} indicates tilting the head joint parallel to the mouth, a la jet-whistle position, \textbf{45°} indicates ordinario, and \textbf{90°} indicates tilting the head joint perpendicular to the mouth, creating aeolian sound. \textbf{2.)} Diamond shaped note heads are used to represent overblowing through the harmonic series of a fundamental. This work only uses the fundamentals \textbf{C} and \textbf{E-flat}.
\endgroup

\begingroup
\textbf{English horn: 1.) Multiphonics} are accompanied with fingering diagrams and approximate pitches in the staff. Not all notated pitches must sound. \textbf{2.) Trill spanners} indicate timbre trills. When combined with glissandi, the timbre trill should persist in the hand as the embouchure controls the glissando.
\endgroup

\begingroup
\textbf{Tuba: 1.) Diads} in this score are always executed by singing the high pitch and playing the low. 
\endgroup

\begingroup
\textbf{Percussion: 1.) The instruments} are a very large gong, a stone roughly 1' x 1' in size, a brake drum, a large floor tom, and a metal Samba whistle. \textbf{2.) The implements} are a hard gong mallet, two hand-sized stones, and two drum sticks. \textbf{3.) The floor tom} should be damped, removing as much resonance from the instrument as possible. \textbf{4.) The Samba whistle} is notated on a two-line staff, wherein the top line indicates the highest note possible on the whistle, and the bottom line indicates the lowest note possible on the whistle. The pitches of notes between the lines should be approximated based on their spatial proximity to the top or bottom line. \textbf{5.) Circular arrow articulations} indicate to draw the implement over the instrument in a circle, completing the circle within the duration of the articulated note.
\endgroup

\begingroup
\textbf{Mezzo-soprano: 1.) The openness of the mouth} is represented by percentage articulations below the staff, wherein \textbf{0\%} indicates fully closed, and \textbf{100\%} indicates as open as possible. In the absence of these articulations, the openness of the mouth is left to the discretion of the interpreter. \textbf{2.) Mouth shape} is indicated using IPA symbols below the staff. Shapes may be interpolated, signaled by a dashed arrow. \textbf{3.) A single-line staff} indicates to audibly breathe through a passage on the rhythm indicated without vibrating the vocal folds.
\endgroup

\begingroup
\textbf{Strings: 1.) The bow speed indications} in this score are \textbf{extra fast bow, or XFB,} which indicates almost an irregular tremolo, moving the bow as quickly and with as full strokes as possible,  \textbf{fast bow, or FB,} which indicates to bow at flautando speed, though not necessarily sul tasto, \textbf{normale bow, or NB,}  which indicates normale bow speed, and \textbf{extra slow bow, or XSB,} which indicates to bow as slowly as possible, generating scratch tone at higher bow pressures. \textbf{2.) Spectral microtones} are indicated by a cent-deviation articulation printed above an equally tempered note. In the absence of electric tuners, approximations of these deviations are acceptable. \textbf{3.) Finger pressure of the left hand} is indicated by note head shape, wherein traditional note heads indicate a fully closed string, triangle-shaped note heads indicate a pressure half-way between harmonic pressure and fully closing the string, and diamond-shaped note heads indicate to touch the notated pitch with pressure as if playing a harmonic, whether a harmonic sounds or not. \textbf{4.) Molto sul ponticello, or MSP} indicates to play with half of the bow hair directly on the bridge and half of the hair on the string, \textbf{sul ponticello, or SP} indicates to bow near the bridge, \textbf{sul tasto, or ST} indicates to bow above the edge of the fingerboard, \textbf{molto sul tasto, or MST} indicates to bow as close to the fingers as possible, \textbf{col legno tratto, or CLT} indicates to bow with the wood, and \textbf{Crine} cancels col legno tratto. {4.) Dietro ponticello, or DP} indicates to play between the bridge and the tailpiece, on the wrapping. When playing this technique, interpreters read a four-line staff wherein the top line represents string I, the next line represents string II, etc.
\endgroup

\begingroup
\textbf{Scordatura: 1.) The first cello} should detune the fourth string down a whole-step to \textbf{B-flat}. \textbf{2.) the second cello} should detune the fourth string down a minor-third to \textbf{A}. \textbf{3.)} Passages performed on the detuned strings are transposed to the physical playing position on the string rather than the actual sounding pitch.
\endgroup

\end{document}