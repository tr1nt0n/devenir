\documentclass[11pt]{article}
\usepackage{fontspec}
\usepackage[utf8]{inputenc}
\setmainfont{Bodoni 72 Book}
\usepackage[paperwidth=11in,paperheight=17in,margin=1in,headheight=0.0in,footskip=0.5in,includehead,includefoot,portrait]{geometry}
\usepackage[absolute]{textpos}
\TPGrid[0.5in, 0.25in]{23}{24}
\parindent=0pt
\parskip=12pt
\usepackage{nopageno}
\usepackage{graphicx}
\graphicspath{ {./images/} }
\usepackage{amsmath}
\usepackage{tikz}
\newcommand*\circled[1]{\tikz[baseline=(char.base)]{
            \node[shape=circle,draw,inner sep=1pt] (char) {#1};}}

\begin{document}

\vspace*{4\baselineskip}

\begingroup
\begin{center}
\huge NOTES FOR THE INTERPRETERS
\end{center}
\endgroup

\begingroup
\textbf{General: 1.)} This work is to be played behind a curtain, with each instrument amplified. The listeners will be in the center of a nine-speaker ellipse. Each speaker will receive the signal of a single instrument. The ordering of the instruments, starting with the frontmost speaker and rotating clockwise, are as follows: mezzo-soprano, second cello, bass flute, English horn, percussion, tuba, violin, flute, first cello. \textbf{2.)} The poem following this page should be read, amplified and from behind the curtain, by the mezzo-soprano, as it is projected before the listeners. If the work is performed between the dates of December 21st and March 19th, the poem should be read before the music. If the work is performed between the dates of March 20th and December 20th, the poem should be read after the music. \textbf{3.)}  Dynamics in this score are effort dynamics, representing the physical force behind an action rather than the sounding dynamic. \textbf{4.)} Dashed arrows above the staff indicate a gradual transition from one technique or tempo to another. \textbf{5.)}  Stem tremoli are to be performed as quickly as possible, and do not represent a subdivision of a note.
\endgroup



\end{document}